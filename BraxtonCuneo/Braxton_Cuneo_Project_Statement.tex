\documentclass[letterpaper,10pt,titlepage]{article}

\usepackage{graphicx}                                        
\usepackage{amssymb}                                         
\usepackage{amsmath}                                         
\usepackage{amsthm}                                          

\usepackage{alltt}                                           
\usepackage{float}
\usepackage{color}
\usepackage{url}

\usepackage{balance}
%\usepackage[TABBOTCAP, tight]{subfigure}
\usepackage{enumitem}
%\usepackage{pstricks, pst-node}

\usepackage[margin=0.75in]{geometry}
\geometry{textheight=8.5in, textwidth=6in}

%random comment

\newcommand{\cred}[1]{{\color{red}#1}}
\newcommand{\cblue}[1]{{\color{blue}#1}}

\usepackage{hyperref}
\usepackage{geometry}
\usepackage{listings}

\def\name{Braxton Cuneo}


\def\name{Braxton Cuneo}

%% The following metadata will show up in the PDF properties
\hypersetup{
  colorlinks = true,
  urlcolor = black,
  pdfauthor = {Braxton Cuneo},
  pdfkeywords = {cs461},
  pdftitle = {3D Painting Group},
  pdfsubject = {CS 461 Problem Statement},
  pdfpagemode = UseNone
}

\renewcommand{\familydefault}{\rmdefault}

\begin{document}

\begingroup
\center
~\\~\\~\\~\\~\\
\LARGE{Problem Statement}\\~\\~\\
\large{CS 461}\\~\\
\large{Fall 2017}\\~\\
\large{Braxton Cuneo}\\~\\~\\~\\~\\~\\
\large{Abstract}\\~\\
\endgroup

\begingroup
The aim of this project is to develop an immersive virtual environment which enables one to be able to create three-dimensional content via three-dimensional gestures. It is noted one must solve a number of consequent problems, including establishing sensible methods of viewing, modifying, and simulating such an environment. Given the relative lack of pre-existing implementations of such an environment, it is concluded that part of its development would necessarily require iterations of trial and error in order to ensure effectiveness. Nonetheless, it is established that pieces hardware capable of displaying the environment, sensing a user's gestural motions, and simulating the behavior and appearance of this environment are needed. Furthermore, it is noted that determining the basic element which the environment is composed of is a large factor in the strengths and weaknesses of the final product. Finally, it is concluded that the project must be held to a standard of frame rate and spatial accuracy to preserve the user's well being and that the user should be capable of modifying and saving the environment for it to count as fulfilling the basic spirit of the problem.
\endgroup

\newpage
\section{Problem Definition}

Many of the means humans have had to express ideas throughout the centuries have grown with the rise of technology into virtual counterparts. Casting aside pencils, writers have taken up text editors. Dropping paint brushes, two-dimensional artists can pick up a stylus on a myriad of image manipulation applications. The sculptors of the world, however, remain in transition. While many model editing applications exist, few have captured the spatial aspect of sculpting in a realistic way, and those that have perhaps leave a lot to be desired. The aim of this project is to bring the world closer to a complete virtual sculpting place, creating a cohesive and believable environment for content creators to manipulate three dimensional forms.

With this goal, there are a number of implied, prerequisite goals. In order for such a virtual environment to be believable, as one would need to be able to see it. Moreover, one should be able to move around this environment with little to no delay in response by whatever was showing this environment.

One must additionally consider the ability one has to manipulate such an environment. Content creation cannot happen in an environment which cannot be altered. Therefore, methods of altering this environment in some intuitive way must be established and implemented. Furthermore, this manipulation of such a three-dimensional world must occur through three-dimensional action as well, as sculpting is not simply about seeing with depth, it is about working in it as well. 

Lastly, there is the matter of processing and latency. Establishing a ruleset for how an environment looks and behaves can mean little if there is no practical means of simulating such an environment. Additionally, to ensure immersion in an environment as well as accurate representation of how that environment develops over time, the ability to quickly update the environment one sees based off of their changes in perspective as well as the actions they take is key.


\section{Proposed Solution}

There is no predetermined way of solving the above problems “right.” The creation of creative virtual environments, particularly three-dimensional ones, is still in active development. Moreover, while products such as Google’s Tilt Brush may come close to solving the above problem, they are not exactly what this team aims to do. As such, it is proposed that this team takes a look both at what has been done and what could be done in the field of three-dimensional creative spaces and to composite a solution to this problem over an iterative process. In frontiers such as these, sticking to a rigid plan can lead to poor results, as unforeseen issues may make these plans less effective than they initially appear.

Thus, a process with some trial and error will be required to investigate the possible space of solutions at our disposal so that the correct course of development can be taken. Agile software development would serve to structure this iterative process into a manageable and measurable effort. Furthermore, starting with simpler hardware such as monitors before moving to more advanced devices such as VR goggles would be advisable.

This is not to say that the only factor in solving this problem is improvisation, there are a number of strong guidelines established by industries such as VR which this team would be remiss to ignore. For instance, nausea is a strong factor for many people which prevents them from using VR for extended periods of time. It has been found that higher frame rates as well as more accurate positional data reduces the discrepancy between what users see and feel, decreasing the overall effects of nausea. Therefore, this team should be expected to hold itself to a minimal level of accuracy and frame rate in order to avoid the majority of these effects.

Additionally, common sense dictates a number of first steps this team should take in constructing this environment. In order to convey three-dimensional gestures, devices capable of detecting such gestures (for instance, Wii controllers or Vive controllers) will need to be used. If the user of this environment needs to feel as though they are in that environment, VR devices will need to be used. To meet the computational requirements of rendering a dynamic environment, device with sufficient graphical power must be used in conjunction with the software the team develops.

As for the environment itself, one may not be able to predict what it would optimally look like, but one can certainly establish how a user may affect it. On the most basic level, in order to work with three-dimensional elements, one should expect to be able to do three things: create these elements, modify these elements, and destroy these elements. As for what constitutes an “element” of a scene, that is still to be determined. In their broadest categories, the two types of elements one can work with are triangles and voxels, where triangles make up the contours of the boundaries which envelope a volume and voxels make up the blocks in a three-dimensional grid that fill a volume. Both have benefits and drawbacks. For volumetric effects, gradations, additive and subtractive operations, and collision detection, voxels are easier to work with. For transformations, rotations, scaling, skeleton-based movement, and data compression (should one decide to save an environment’s state), triangles are more convenient.

After determining the hardware and scene element types this project will use, this team will refine their operation in conjunction with the creative environment it will construct, hopefully to the fulfillment of the above problem.


\section{Performance Metrics}

While a certain aspect of this project is necessarily explorative in nature, there are a number of metrics which, if aimed for, should act as good goals regardless of final implementations. Firstly, the measurement accuracy of both the gestural devices and headsets used should at least on the level of millimeters. Secondly, a minimum frame rate of the displays used should be 50 frames per second. Additionally, the user should at least be able to perform the three basic operations detailed above on the virtual environment: adding elements, modifying elements, and removing elements.  Lastly, the user should be able to save the state of the environment to a file and be able to recreate that state given that same file.

Given the above metrics, one can hopefully guarantee that the user experiences minimal discomfort when interacting with the program this team creates and that this program meaningfully fulfills the role of a content creation environment.


\end{document}
