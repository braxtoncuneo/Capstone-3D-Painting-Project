\documentclass[onecolumn, draftclsnofoot,10pt, compsoc]{IEEEtran}
\usepackage{graphicx}
\usepackage{url}
\usepackage{setspace}

\usepackage{geometry}
\geometry{textheight=9.5in, textwidth=7in}

% 1. Fill in these details
\def \CapstoneTeamName{			PolyVox}
\def \CapstoneTeamNumber{		61}
\def \GroupMemberOne{			Chris Bokkam}
\def \GroupMemberTwo{			Richard Cunard}
\def \GroupMemberThree{			Braxton Cuneo}
\def \CapstoneProjectName{		VR Painting Project}
\def \CapstoneSponsorCompany{		EECS}
\def \CapstoneSponsorPersonOne{		Dr. Kirstin Winters}
\def \CapstoneSponsorPersonTwo{		Dr. Mike Bailey}

% 2. Uncomment the appropriate line below so that the document type works
\def \DocType{		Problem Statement
				%Requirements Document
				%Technology Review
				%Design Document
				%Progress Report
				}
			
\newcommand{\NameSigPair}[1]{\par
\makebox[2.75in][r]{#1} \hfil 	\makebox[3.25in]{\makebox[2.25in]{\hrulefill} \hfill		\makebox[.75in]{\hrulefill}}
\par\vspace{-12pt} \textit{\tiny\noindent
\makebox[2.75in]{} \hfil		\makebox[3.25in]{\makebox[2.25in][r]{Signature} \hfill	\makebox[.75in][r]{Date}}}}
% 3. If the document is not to be signed, uncomment the RENEWcommand below
%\renewcommand{\NameSigPair}[1]{#1}

%%%%%%%%%%%%%%%%%%%%%%%%%%%%%%%%%%%%%%%
\begin{document}
\begin{titlepage}
    \pagenumbering{gobble}
    \begin{singlespace}
    	\includegraphics[height=4cm]{coe_v_spot1}
        \hfill 
        % 4. If you have a logo, use this includegraphics command to put it on the coversheet.
        %\includegraphics[height=4cm]{CompanyLogo}   
        \par\vspace{.2in}
        \centering
        \scshape{
            \huge CS Capstone \DocType \par
            {\large\today}\par
            \vspace{.5in}
            \textbf{\Huge\CapstoneProjectName}\par
            \vfill
            {\large Prepared for}\par
            \Huge \CapstoneSponsorCompany\par
            \vspace{5pt}
            {\Large\NameSigPair{\CapstoneSponsorPersonOne}\par}
	    {\Large\NameSigPair{\CapstoneSponsorPersonTwo}\par}
            {\large Prepared by }\par
            Group\CapstoneTeamNumber\par
            % 5. comment out the line below this one if you do not wish to name your team
            \CapstoneTeamName\par 
            \vspace{5pt}
            {\Large
                \NameSigPair{\GroupMemberOne}\par
                \NameSigPair{\GroupMemberTwo}\par
                \NameSigPair{\GroupMemberThree}\par
            }
            \vspace{20pt}
        }
        \begin{abstract}
        % 6. Fill in your abstract    
        	 The goal of this project is to design and implement a virtual reality (VR) system that allows a user to create three dimensional art in a virtual space.
		 Our project requires the construction of a program that supports user generated geometry and color control through use of a movement based interface.
		 The purpose of this is to enhance creativity by leveraging the technology behind virtual reality. 
		 This software will be designed for artists and presenters to gain an in depth representation of their creations and creation process. 
		 The project will be designed with specifications that outline usability, necessary features, technical requirements and environment controls. 
		 The finalization of the software will have the ability to for users to paint, modify and transform their own virtual reality space.
        \end{abstract}     
    \end{singlespace}
\end{titlepage}
\newpage
\pagenumbering{arabic}
\tableofcontents
% 7. uncomment this (if applicable). Consider adding a page break.
%\listoffigures
%\listoftables
\clearpage

% 8. now you write!
\section{Problem Statement}
This project is the design and construction of a VR system that allows the user to create three dimensional objects in virtual space using a motion control device.
This system needs to be capable of detecting and measuring user movement as a means of creating and manipulating these forms.
Additionally, this system must also allow for the changing of object coloration (painting) through the use of motion controls. 

One of the first obstacles to overcome in this project is creating an interface for user interaction with this virtual environment.
Virtual reality is a rather young medium, and so reference material concerning the construction of such an interface is scarce.
This means that the team carrying out this project will have to build to the UI independent from other frameworks.
This necessitates extensive development and testing in order to create an effective user experience.

Additionally, it should be noted that a poorly designed VR system can cause nausea, dizziness, and physical pain to end users.
Avoiding these pitfalls necessitates the system to be designed and held to a set of technical standards which minimizes the factors contributing to these effects.
This also requires the underlying program to be able to operate according to a predetermined set of performance metrics, on hardware meeting a set of minimum requirements. 

One of the primary challenges faced with this project is determining where it begins and ends.
Preliminary project documents have suggested that the team has some latitude in determining the specific features of the final project.
Whether this entails the ability to perform precision sculpting, or give the user an extensive library of ‘paints’ to choose from when coloring objects, or even creating a system to allow for user created content, needs to be determined for the project to proceed.
The project will have to define the following: creating, modifying and destroying 3D virtual geometry; displaying these virtual objects to the user in three dimensions; saving and loading said geometry, and allow the user to perform all of these actions using some form of motion control. All of these concepts must be clearly stated and solved by this project.

\section{Solutions}
With level of complexity and required workload to construct such a program, the team will use a pre-existing game engine in conjunction with a VR software development kit (SDK) to construct our program.
As such, determining what specific tools to use will be among the first steps taken during the project.
In turn, these tools will be used to design and construct the motion control interface.
Designing the interface will itself be a major aspect of the project.
VR user experience (UX) design is still in its infancy, and will require a certain degree of trial and error.
To overcome this, the team will study existing VR applications and their interfaces, and employ a flexible development cycle to allow for experimentation and modification of the system as needed.
Likely the most significant technical challenge will be optimizing the program such to avoid user discomfort.
The primary means by which this will be done is through maintaining sufficient code and design standards to ensure that the program will run effectively on hardware designated as sufficient. 

\section{Performance Metrics}
In order to complete this project effectively, a number of requirements must met pertaining to the abilities of the user in the space, the limitations of the environment, and the quality of the environment’s representation.

The baseline capabilities the user should expect to have in this creative space fall under three main categories.
The first of these is the manipulation of the environment, which includes the creation, modification, and removal of elements.
Secondly, the user should be able to navigate the 3D presentation environment while using a VR headset.
Finally, the user should be able to save the state of the environment, which may later be used by the program to reconstruct this state.
These metrics are to be upheld primarily to meet the basic sentiment established by the problem.

For the environment itself, one particular limitation that must be met is that the elements it contains must always form legal solids.
This metric serves to ensure easier inclusion of useful features which could be part of the final project, such as enabling faster boolean operations between geometric forms.
Furthermore, this limitation has the added benefit of making the state of the virtual space more easily convertible to a 3D-printable file format.

As for the representation of the environment to the user, the accuracy of the user’s view of the environment should match some minimum standards of quality to minimize the negative physiological effects associated with unrealistic/unintuitive VR.
These standards include a maximum latency in updates to the user’s relative position in the environment, a minimum number of frames produced per second by the system, and a minimum level of accuracy for the positional and rotational measurements made by the headset and motion controls.


\end{document}
